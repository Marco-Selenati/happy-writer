\documentclass[a4paper, 11pt]{article}

% packages
\usepackage{fontspec}
\usepackage{textcomp}
% syntax highlighting
\usepackage{minted}
% Auflistungen
\usepackage[ampersand]{easylist}
% Verbesserungen zum style
\usepackage{amsmath, amssymb, amsthm}
% Frage Antwort
\usepackage{dramatist}
% hyperlinks
\usepackage{hyperref}
% Sprachunterstützung
\usepackage[nswissgerman]{babel}
% schönere font
\usepackage{unicode-math}
% hier setzen wir die Font welche bei den Codeauflistungen verwendet werden soll
\setmonofont[
  Contextuals={Alternate}
]{Fira Code}
\usepackage{lmodern}
% header und footer
\usepackage{fancyhdr}
% hier definieren wir unseren eigenen header/footer style
\lhead{}
\chead{}
\rhead{}
\lfoot{}
\cfoot{\thepage}
\rfoot{}

% Setzt die default Formatierung für die Javasprache
\setminted[java]{
frame=leftline,
framesep=8mm,
numbersep=12pt,
numbersep=5pt,
fontsize=\small,
baselinestretch=1.5,
tabsize=4,
gobble=0,
linenos
}

% lange Wörter auseinander brechen
\hyphenation{
	datenbank-gestützten
	Schreib-waren-handlung
}
% dokument informationen/title
\title{Dokumentation Projektarbeit `HappyWriter'}
\author{Marco Selenati\\
	IFZ-624-002\\
	Stiftung Wirtschaftsinformatikschule Schweiz,\\
	Hohlstrasse 535,\\
	Zürich,\\
	Schweiz\\
	marcoselenati@gmail.com}
\date{\today}

\begin{document}
% Titelseite
\maketitle
\clearpage

% Inhaltsverzeichnis
\pagenumbering{roman}
\tableofcontents
% nichts anderes sollte auf der Inhaltsseite sein
% darum beenden wir hier die Seite
\clearpage
% Anfang Inhalt
\pagenumbering{arabic}
\pagestyle{fancy}

\begin{abstract}

	Erstellung eines datenbankgestützten Online-Shops für eine Schreibwarenhandlung, mit einem Verwaltungssystem für Inhaltsartikel.

\end{abstract}

\section{Einleitung}

Bei diesem Projekt geht es darum, für den Schreibwarenhandel `HappyWriter' einen datenbankgestützten Online-Shop zu erstellen.
Ich bin Marco Selenati, besuche das 4 Semester der Wirtschaftsinformatikschule Zürich und bin einziger Projektteilnehmer.
Diese Projektbeschreibung wird Ihnen aufzeigen wie ich diesen Webshop erstellt habe.

\section{Problemstellung}

Ziel dieses Projektes ist es, innerhalb einer Woche dem Unternehmen einen funktionierenden Webshop zu Verfügung zu stellen.
In diesem ist auch ein System für das Verwalten von Inhaltsartikeln inbegriffen.
Dieser Webshop eröffnet dem Unternehmen einen neuen Verkaufskanal.

\section{Anforderungen}

Die Projektanforderungen, welche aus den Aufgabeblätter\cite{Aufgabeblaetter} sind, sind wie folgt.

\begin{easylist}[itemize]
	& Project Wonder
	& SQL Skript einrichtung der DB
	& Dokumentation als PDF
	&& Betriebshandbuch
	&& Hilfestellungen
	& Ein in Eclipse importierbares Projekt
	& Das vorgegebene ERD benutzen
\end{easylist}

\section{Projektdurchführung}

\subsection{Planung}

\begin{easylist}
	& Erstellen des Konzeptes
	&& Klassen
	&& Administrationsbereich
	&& Tests

	& Projektumgebung herstellen
	&& Erstellen des git Repository

	& Erstellen des EOModels
	&& Nach gegebenen ERD
	&& SQL kommentieren
	&& Seed (Test) Daten

	& Funktionalitäten erstellen
	&& Der normalbenutzer Seiten
	&& Der Administrations Seiten

	& Seite testen
	& Die Seiten Stylen
	& Form validierung
	& Seite testen
	& Projekt dokumentation schreiben

\end{easylist}

\subsection{Probleme}

\subsubsection{Inhalte anzeigen}

\subsubsection{Checkbox array}

\subsubsection{Validation}

\subsubsection{Component Component variablen}

\subsubsection{Formular abbruch}

\section{Projektergebnisse}

\subsection{Testkonzept}

\subsection{Ordnerstruktur}

\section{Projektbewertung}

\subsection{Schlussfolgerung}

\subsection{Reflexion}

Die Arbeit an diesem Projekt war sehr interessant.
Ich konnte erleben, wie einfach WOnder es einem macht, eine Applikation zu bauen.
Dies ist aber nur möglich, wenn man objektorientiert denkt und die Möglichkeiten von WOnder kennt.
Ich muss beim nächsten Projekt die Transactions mehr im Kopf haben.

\subsection{Zukunft}

\addcontentsline{toc}{section}{Literatur}
\begin{thebibliography}{9}
\bibitem{Aufgabeblaetter}
Leistungsbeurteilung Projektarbeit
Datenbanken in Web-Auftritt einbinden (151)
Dokumentnachweis: B14-151\_LB-Projektarbeit\_Aufgabenstellung\_18-06\_V004.docx

\bibitem{Learning_the_Wonders}
Learning the Wonders
Markus Ruggiero
rucotec GmbH, Switzerland; 1 edition (September 20, 2013)
ASIN B00FCDHDAA

\end{thebibliography}

\section{Anhang}

\subsection{Benutzte Ressourcen}

\subsection{Zeiterfassung}

\subsection{Informationsbeschaffung}

\begin{drama}
	\Character{Marco Selenati}{ms}
	\Character{Markus Ruggiero}{dozent}
	\Character{Internet}{internet}
	\Character{Herr Bärlocher}{lucas}
	\Character{Frau Fusco}{fusco}
	\Character{Herr Standhart}{standhard}

\msspeaks: Kann ich die Dokumentation solange sie am Schluss PDF ist aus allem generieren lassen was ich will?

\dozentspeaks: Ja, Sie können es aus Markdown Word oder sogar aus einem selbst geschriebenen Tool generieren lassen wenn Sie wollen.

\msspeaks: Auf dem vorgegebenen ERD steht, dass bei der Konfigurationstabelle ein Attribut mit dem Namen inhalt\_d sein soll. Muss ich das genau so machen?

\dozentspeaks: Nein, das sollte inhalt\_id sein. Das muss ein Tippfehler sein.

\msspeaks: Die Bestellübersichtsseite ist nirgends im Text vorhanden. Wo kommt diese hin?

\dozentspeaks: Die Bestellübersichtsseite ist eine bessere Version der Mainseite. Da die Bestellübersichtsseite eine bessere Warenkorbauflistung hat.

\msspeaks: Kann ich die Bilder der Inhalte bei der Auswahl der Inhalte anzeigen?

\dozentspeaks: Wäre gut, wenn es so wäre.

\msspeaks: Muss ich angeben dass ich die gegebenen Bilder benutze?

\dozentspeaks: Ja, in der Dokumentation reicht. Sie können es schon auf der Seite angeben, aber es ist ok, wenn ohne.

\msspeaks: Darf ich die Namen im UML umbenennen?

\dozentspeaks: Ja.

\msspeaks: Darf ich das ganze anders gestalten wie es im UML steht?

\dozentspeaks: Ja, dass UML ist nur ein Vorschlag.

\msspeaks: Wie benutze ich Checkbox matrix

\internetspeaks: \href{https://github.com/wocommunity/wonder/blob/4d7f6bf9236c3005359101d6f3c9e6224d47750e/Frameworks/Core/JavaWOExtensions/Sources/com/webobjects/woextensions/WOCheckboxMatrix.java}{WOnder dokumentation}

\msspeaks: Wie kann ich nur Buchstaben im Namen und Vornamen haben?

\internetspeaks: \href{https://stackoverflow.com/questions/23415795/how-to-only-allow-text-in-parsely-js-validation}{stackoverflow parsley}

\msspeaks: Wissen Sie, wie man CSS einbinden soll?

\lucasspeaks: ERXStyleSheet

\msspeaks: Wissen Sie, wie man JavaScript einbindet?

\lucasspeaks: ERX JavaScript

\msspeaks: Wie kann ich ein Design für das Aussehen der Webseite erstellen?

\lucasspeaks: draw io.

\msspeaks: Können Sie sich meine Applikation ansehen und mir sagen, ob Sie Probleme sehen?

\lucasspeaks:
    Vorname, Nachname sollte nur Buchstaben haben
    Telefon sollte nur Nummern und + () haben

\msspeaks: Wissen Sie, wie ich die Checkboxenwerte in ein Array reinbekomme?

\standhardspeaks: Ich habe einfach Checkbox Matrix benutzt.

\msspeaks: Wollen Sie sich meine Applikation ansehen?

\fuscospeaks: Ja
    Artikel und Inhalt sind verwirrend. da Inhalte auch Artikel sind. Inhaltsartikel anstatt Inhalt.
    Admin Login sollte auf den CMS Seiten nicht angezeigt werden.

\end{drama}

\begin{minted}{java}
	private Artikel artikel;
	/**
	* Der inhalt in der jetzigen Iteration
	*/
	private Inhalt inhaltLoopVar;
	
	/**
	 * Die Inhalte welche auf der Seite ausgewählt worden
	 * sind, sind in diesem array drin, bei der submit Methode
	 * 
	 */
	private NSMutableArray<Inhalt> inhalteSelections;
	\end{minted}
\end{document}
